\documentclass{beamer}

\pdfmapfile{+sansmathaccent.map}


\mode<presentation>
{
  \usetheme{Warsaw} % or try Darmstadt, Madrid, Warsaw, Rochester, CambridgeUS, ...
  \usecolortheme{seahorse} % or try seahorse, beaver, crane, wolverine, ...
  \usefonttheme{serif}  % or try serif, structurebold, ...
  \setbeamertemplate{navigation symbols}{}
  \setbeamertemplate{caption}[numbered]
} 


%%%%%%%%%%%%%%%%%%%%%%%%%%%%
% itemize settings

\definecolor{mypink}{RGB}{255, 30, 80}
\definecolor{mydarkblue}{RGB}{60, 160, 255}
\definecolor{myblue}{RGB}{240, 240, 255}
\definecolor{mygreen}{RGB}{0, 200, 0}
\definecolor{mygreen2}{RGB}{245, 255, 230}
\definecolor{mygray}{gray}{0.8}

\setbeamertemplate{itemize items}[default]

\setbeamertemplate{itemize item}{\color{mygreen}$\blacksquare$}
\setbeamertemplate{itemize subitem}{\color{mydarkblue}$\blacktriangleright$}
\setbeamertemplate{itemize subsubitem}{\color{mygray}$\blacksquare$}



\setbeamercolor{palette quaternary}{fg=white,bg=mydarkblue}
\setbeamercolor{titlelike}{parent=palette quaternary}

\setbeamercolor{palette quaternary2}{fg=black,bg=myblue}
\setbeamercolor{frametitle}{parent=palette quaternary2}



\setbeamerfont{frametitle}{size=\Large,series=\scshape}
\setbeamerfont{framesubtitle}{size=\normalsize,series=\upshape}





%%%%%%%%%%%%%%%%%%%%%%%%%%%%
% block settings

\setbeamercolor{block title}{bg=red!30,fg=black}

\setbeamercolor*{block title example}{bg=mygreen!40!white,fg=black}

\setbeamercolor*{block body example}{fg= black,
bg= mygreen2}


%%%%%%%%%%%%%%%%%%%%%%%%%%%%
% URL settings
\hypersetup{
    colorlinks=false,
    linkcolor=blue,
    filecolor=blue,      
    urlcolor=blue,
}

%%%%%%%%%%%%%%%%%%%%%%%%%%

\renewcommand{\familydefault}{\rmdefault}

\usepackage{amsmath}
\usepackage{mathtools}


\usepackage{subcaption}

\newcommand{\bo}[1] {\mathbf{#1}}
\newcommand{\R} {\mathbb{R}}
\DeclareMathOperator*{\argmin}{arg\,min}


%%%%%%%%%%%%%%%%%%%%%%%%%%%%
% code settings

\usepackage{listings}
\usepackage{color}
% \definecolor{mygreen}{rgb}{0,0.6,0}
% \definecolor{mygray}{rgb}{0.5,0.5,0.5}
\definecolor{mymauve}{rgb}{0.58,0,0.82}
\lstset{ 
  backgroundcolor=\color{white},   % choose the background color; you must add \usepackage{color} or \usepackage{xcolor}; should come as last argument
  basicstyle=\footnotesize,        % the size of the fonts that are used for the code
  breakatwhitespace=false,         % sets if automatic breaks should only happen at whitespace
  breaklines=true,                 % sets automatic line breaking
  captionpos=b,                    % sets the caption-position to bottom
  commentstyle=\color{mygreen},    % comment style
  deletekeywords={...},            % if you want to delete keywords from the given language
  escapeinside={\%*}{*)},          % if you want to add LaTeX within your code
  extendedchars=true,              % lets you use non-ASCII characters; for 8-bits encodings only, does not work with UTF-8
  firstnumber=0000,                % start line enumeration with line 0000
  frame=single,	                   % adds a frame around the code
  keepspaces=true,                 % keeps spaces in text, useful for keeping indentation of code (possibly needs columns=flexible)
  keywordstyle=\color{blue},       % keyword style
  language=Octave,                 % the language of the code
  morekeywords={*,...},            % if you want to add more keywords to the set
  numbers=left,                    % where to put the line-numbers; possible values are (none, left, right)
  numbersep=5pt,                   % how far the line-numbers are from the code
  numberstyle=\tiny\color{mygray}, % the style that is used for the line-numbers
  rulecolor=\color{black},         % if not set, the frame-color may be changed on line-breaks within not-black text (e.g. comments (green here))
  showspaces=false,                % show spaces everywhere adding particular underscores; it overrides 'showstringspaces'
  showstringspaces=false,          % underline spaces within strings only
  showtabs=false,                  % show tabs within strings adding particular underscores
  stepnumber=2,                    % the step between two line-numbers. If it's 1, each line will be numbered
  stringstyle=\color{mymauve},     % string literal style
  tabsize=2,	                   % sets default tabsize to 2 spaces
  title=\lstname                   % show the filename of files included with \lstinputlisting; also try caption instead of title
}

%%%%%%%%%%%%%%%%%%%%%%%%%%%%
% tikz settings

\usepackage{tikz}
\tikzset{every picture/.style={line width=0.75pt}}

%%%%%%%%%%%%%%%%%%%%%%%%%%%%

\usepackage{qrcode}



\title{Laplace Transform and Transfer Functions}
\subtitle{Control Theory, Tutorial 3}
% \author{by Sergei Savin}
\centering
\date{\mydate}



\begin{document}
\maketitle


\begin{frame}{Content}

\begin{itemize}
\item Mechanical systems with inputs
\item Electric motor current eq.
\item Electric motor electro-mechanical eq.
\item Stability
\item Read more
\end{itemize}

\end{frame}




\begin{frame}{Mechanical systems with inputs (1)}
% \framesubtitle{O}
\begin{flushleft}

Previously we have seen a simple spring-damper system:

\begin{equation}
    m \ddot x + \mu \dot x + c x = 0
\end{equation}

where $m$, $\mu$, $c$ are mass, damping coefficient and stiffness coefficient, and $x$ is the position of the mass.

\bigskip

We can add external force $u$, to arrive at the following equation:

\begin{equation}
    m \ddot x + \mu \dot x + c x = u
\end{equation}

\end{flushleft}
\end{frame}


\begin{frame}{Mechanical systems with inputs (2)}
% \framesubtitle{O}
\begin{flushleft}

We can use Laplace transform to change $\dot x$ to $s x$, and with that original eq. $m \ddot x + \mu \dot x + c x = u$ becomes:

\begin{equation}
    m s^2 x + \mu s x + c x = u
\end{equation}

\begin{equation}
    x = \frac{1}{m s^2 + \mu s + c } u
\end{equation}

Thus we found the transfer function of the system:

\begin{equation}
    W(s) = \frac{1}{m s^2 + \mu s + c }
\end{equation}

\end{flushleft}
\end{frame}






\begin{frame}{Electric motor current eq.}
% \framesubtitle{O}
\begin{flushleft}

Dynamics of the current in the windings of an electric motor can be described with the following eq.

\begin{equation}
    L \frac{di}{dt} + R i + C_e \omega = u
\end{equation}

where $L$, $R$, $C_e$, $\omega$, and $u$ are inductance, resistance, back-EMF (back electromotive force) coefficient, angular velocity of the motor's shaft, and input voltage.

\bigskip

We can transfer this with Laplace transform:

\begin{equation}
    L s i + R i + C_e \omega = u
\end{equation}

And we find its transfer function representation:

\begin{equation}
    i = \frac{1}{L s + R} u
\end{equation}

Notice that the TF from voltage to current is independent from $C_e \omega$.

\end{flushleft}
\end{frame}




\begin{frame}{Electric motor electro-mechanical eq. (1)}
% \framesubtitle{O}
\begin{flushleft}

Full dynamics equations of a DS motor are

\begin{equation}
    \begin{cases}
    L \frac{di}{dt} + R i + C_e \omega = u \\
    J \frac{d \omega}{dt} = C_\tau i
    \end{cases}
\end{equation}

where $C_\tau$ and $J$ are torque coef. and apparent inertial coef. of the rotor's shaft.

\bigskip

We can transfer this with Laplace transform:

\begin{equation}
    \begin{cases}
    L s i + R i + C_e \omega = u \\
    J s \omega = C_\tau i
    \end{cases}
\end{equation}

\end{flushleft}
\end{frame}





\begin{frame}{Electric motor electro-mechanical eq. (2)}
% \framesubtitle{O}
\begin{flushleft}

We find transfer function from voltage to angular velocity.

\begin{equation}
    \begin{cases}
    i = \frac{1}{L s + R} u - \frac{C_e}{L s + R} \omega\\
    J s \omega = C_\tau i
    \end{cases}
\end{equation}

\begin{equation}
    J s \omega = \frac{C_\tau}{L s + R} u - \frac{C_\tau C_e}{L s + R} \omega
\end{equation}

\begin{equation}
    (L s + R) J s \omega = C_\tau u - C_\tau C_e \omega
\end{equation}

\begin{equation}
    (J L s^2 + J R s + C_\tau C_e)  \omega = C_\tau u
\end{equation}

\begin{equation}
    \omega = \frac{C_\tau}{J L s^2 + J R s + C_\tau C_e} u
\end{equation}

\end{flushleft}
\end{frame}



\begin{frame}{Electric motor electro-mechanical eq. (3)}
% \framesubtitle{O}
\begin{flushleft}

Finally, the transfer function is:

\begin{equation}
    W_{\omega, u}(s) = \frac{C_\tau}{J L s^2 + J R s + C_\tau C_e}
\end{equation}

Other TFs for this system are:

\begin{equation}
    W_{\omega, i}(s) = \frac{C_\tau }{J s}
\end{equation}

\begin{equation}
    W_{i, u}(s) = 
    \frac{J s}{J L s^2 + J R s + C_\tau C_e} u
\end{equation}


\end{flushleft}
\end{frame}








\begin{frame}{Stability (1)}
% \framesubtitle{O}
\begin{flushleft}

Observing eq. $m \ddot y + \mu_0 \dot y + c_0 y = 0$ we can tell that it is stable if (sufficient but not necessary condition) $m > 0$, $\mu > 0$, and $c > 0$ - this follows from the physics of the system.

\bigskip

A more principled approach is to find eigenvalues of the linear system. We start by dividing the equation by $m$:

\begin{equation}
    \ddot y + \mu \dot y + c y = 0
\end{equation}

where $\mu = \mu_0 / m$ and $c = c_0 / m$. Defining $x_1 = y$ and $x_2 = \dot y$, the system can be equivalently represented as:

\begin{equation}
    \begin{bmatrix}
    \dot x_1 \\ \dot x_2
    \end{bmatrix}
    =
    \begin{bmatrix}
    0 & 1 \\
    -c & -\mu
    \end{bmatrix} 
    \begin{bmatrix}
    x_1 \\ x_2
    \end{bmatrix}   
\end{equation}


\end{flushleft}
\end{frame}



\begin{frame}{Stability (2)}
% \framesubtitle{O}
\begin{flushleft}

With linear system 
$
    \begin{bmatrix}
    \dot x_1 \\ \dot x_2
    \end{bmatrix}
    =
    \begin{bmatrix}
    0 & 1 \\
    -c & -\mu
    \end{bmatrix} 
    \begin{bmatrix}
    x_1 \\ x_2
    \end{bmatrix}   
$, we need to find its eigenvalues. We know that there is a formula for eigenvalues based on trace and determinant:

\begin{equation}
    \lambda = \frac{T \pm \sqrt{T^2 - 4D} }{2}
\end{equation}

where $T$ is trace and $D$ is the determinant.

\bigskip

In our case $T = -\mu$ and $D = c$, and eigenvalues are:

\begin{equation}
    \lambda = \frac{-\mu \pm \sqrt{\mu^2 - 4c} }{2}
\end{equation}


\end{flushleft}
\end{frame}



\begin{frame}{Stability (3)}
% \framesubtitle{O}
\begin{flushleft}

Lets analyse eigenvalues $\lambda = \frac{-\mu \pm \sqrt{\mu^2 - 4c} }{2}$. We can see that if $\mu \geq 0$ and $c \geq 0$, there are only two scenarios: 

\begin{enumerate}
    \item $\mu^2 - 4c \geq 0$, in which case $\sqrt{\mu^2 - 4c} \leq \mu$, the eigenvalues are purely real and negative.
    \item $\mu^2 - 4c < 0$, in which case $\sqrt{\mu^2 - 4c}$ is a purely imaginary number, the eigenvalues are complex with negative real parts.
\end{enumerate}

If $\mu \geq 0$ and $c = 0$, $\lambda_1 = -\mu$, $\lambda_2 = 0$, hence the system is marginally stable.

\end{flushleft}
\end{frame}



\begin{frame}{Stability (4)}
% \framesubtitle{O}
\begin{flushleft}


If $\mu \geq 0$ and $c < 0$, then $\sqrt{\mu^2 - 4c} \geq \mu$, and eigenvalues are purely real and one of them is positive, the system is unstable. If $\mu < 0$ and $c < 0$ at least one of the eigenvalues is still positive.

\bigskip

If $\mu < 0$ and $c \geq 0$, then again there are only two scenarios: 

\begin{enumerate}
    \item $\mu^2 - 4c \geq 0$, in which case $\sqrt{\mu^2 - 4c} \leq \mu$, the eigenvalues are purely real and positive.
    \item $\mu^2 - 4c < 0$, in which case $\sqrt{\mu^2 - 4c}$ is a purely imaginary number, the eigenvalues are complex with positive real parts.
\end{enumerate}

\begin{definition}
If $\mu \geq 0$ and $c \geq 0$ the system is stable, if $\mu < 0$ or $c < 0$ it is unstable.
\end{definition}

\end{flushleft}
\end{frame}




\begin{frame}{Stability (5)}
% \framesubtitle{O}
\begin{flushleft}

Going back to the eq. $\ddot y + \mu \dot y + c y = 0$ we can write characteristic eq. for it:

\begin{equation}
    k^2 y + \mu k + c = 0
\end{equation}

Its roots are given by the formula:

\begin{equation}
    k = \frac{-\mu \pm \sqrt{\mu^2 - 4c} }{2}
\end{equation}

As we can see, it is exactly the same as the determinant-trace formula.


\end{flushleft}
\end{frame}



\begin{frame}{Stability (6)}
% \framesubtitle{O}
\begin{flushleft}

Now, lets consider the transfer function:

\begin{equation}
    W(s) = \frac{1}{s^2 + \mu s + c }
\end{equation}

Notice that its determinant is exactly the same as the characteristic polynomial. This illustrates the connection between the stability criteria for transfer function and other representations.


\end{flushleft}
\end{frame}






% \begin{frame}{Read more}

% \begin{itemize}
% \item ??

% \item ??

% \end{itemize}

% \end{frame}



\begin{frame}{Thank you!}
\centerline{Lecture slides are available via Moodle.}
\bigskip
\centerline{You can help improve these slides at:}
\centerline{\mygit}
\bigskip
\centerline{Check Moodle for additional links, videos, textbook suggestions.}
\bigskip

\centerline{\textcolor{black}{\qrcode[height=1.6in]{https://github.com/SergeiSa/Control-Theory-Slides-Spring-2022}}}
\end{frame}

\end{document}
