\documentclass{beamer}

\pdfmapfile{+sansmathaccent.map}


\mode<presentation>
{
  \usetheme{Warsaw} % or try Darmstadt, Madrid, Warsaw, Rochester, CambridgeUS, ...
  \usecolortheme{seahorse} % or try seahorse, beaver, crane, wolverine, ...
  \usefonttheme{serif}  % or try serif, structurebold, ...
  \setbeamertemplate{navigation symbols}{}
  \setbeamertemplate{caption}[numbered]
} 


%%%%%%%%%%%%%%%%%%%%%%%%%%%%
% itemize settings

\definecolor{mypink}{RGB}{255, 30, 80}
\definecolor{mydarkblue}{RGB}{60, 160, 255}
\definecolor{myblue}{RGB}{240, 240, 255}
\definecolor{mygreen}{RGB}{0, 200, 0}
\definecolor{mygreen2}{RGB}{245, 255, 230}
\definecolor{mygray}{gray}{0.8}

\setbeamertemplate{itemize items}[default]

\setbeamertemplate{itemize item}{\color{mygreen}$\blacksquare$}
\setbeamertemplate{itemize subitem}{\color{mydarkblue}$\blacktriangleright$}
\setbeamertemplate{itemize subsubitem}{\color{mygray}$\blacksquare$}



\setbeamercolor{palette quaternary}{fg=white,bg=mydarkblue}
\setbeamercolor{titlelike}{parent=palette quaternary}

\setbeamercolor{palette quaternary2}{fg=black,bg=myblue}
\setbeamercolor{frametitle}{parent=palette quaternary2}



\setbeamerfont{frametitle}{size=\Large,series=\scshape}
\setbeamerfont{framesubtitle}{size=\normalsize,series=\upshape}





%%%%%%%%%%%%%%%%%%%%%%%%%%%%
% block settings

\setbeamercolor{block title}{bg=red!30,fg=black}

\setbeamercolor*{block title example}{bg=mygreen!40!white,fg=black}

\setbeamercolor*{block body example}{fg= black,
bg= mygreen2}


%%%%%%%%%%%%%%%%%%%%%%%%%%%%
% URL settings
\hypersetup{
    colorlinks=false,
    linkcolor=blue,
    filecolor=blue,      
    urlcolor=blue,
}

%%%%%%%%%%%%%%%%%%%%%%%%%%

\renewcommand{\familydefault}{\rmdefault}

\usepackage{amsmath}
\usepackage{mathtools}


\usepackage{subcaption}

\newcommand{\bo}[1] {\mathbf{#1}}
\newcommand{\R} {\mathbb{R}}
\DeclareMathOperator*{\argmin}{arg\,min}


%%%%%%%%%%%%%%%%%%%%%%%%%%%%
% code settings

\usepackage{listings}
\usepackage{color}
% \definecolor{mygreen}{rgb}{0,0.6,0}
% \definecolor{mygray}{rgb}{0.5,0.5,0.5}
\definecolor{mymauve}{rgb}{0.58,0,0.82}
\lstset{ 
  backgroundcolor=\color{white},   % choose the background color; you must add \usepackage{color} or \usepackage{xcolor}; should come as last argument
  basicstyle=\footnotesize,        % the size of the fonts that are used for the code
  breakatwhitespace=false,         % sets if automatic breaks should only happen at whitespace
  breaklines=true,                 % sets automatic line breaking
  captionpos=b,                    % sets the caption-position to bottom
  commentstyle=\color{mygreen},    % comment style
  deletekeywords={...},            % if you want to delete keywords from the given language
  escapeinside={\%*}{*)},          % if you want to add LaTeX within your code
  extendedchars=true,              % lets you use non-ASCII characters; for 8-bits encodings only, does not work with UTF-8
  firstnumber=0000,                % start line enumeration with line 0000
  frame=single,	                   % adds a frame around the code
  keepspaces=true,                 % keeps spaces in text, useful for keeping indentation of code (possibly needs columns=flexible)
  keywordstyle=\color{blue},       % keyword style
  language=Octave,                 % the language of the code
  morekeywords={*,...},            % if you want to add more keywords to the set
  numbers=left,                    % where to put the line-numbers; possible values are (none, left, right)
  numbersep=5pt,                   % how far the line-numbers are from the code
  numberstyle=\tiny\color{mygray}, % the style that is used for the line-numbers
  rulecolor=\color{black},         % if not set, the frame-color may be changed on line-breaks within not-black text (e.g. comments (green here))
  showspaces=false,                % show spaces everywhere adding particular underscores; it overrides 'showstringspaces'
  showstringspaces=false,          % underline spaces within strings only
  showtabs=false,                  % show tabs within strings adding particular underscores
  stepnumber=2,                    % the step between two line-numbers. If it's 1, each line will be numbered
  stringstyle=\color{mymauve},     % string literal style
  tabsize=2,	                   % sets default tabsize to 2 spaces
  title=\lstname                   % show the filename of files included with \lstinputlisting; also try caption instead of title
}

%%%%%%%%%%%%%%%%%%%%%%%%%%%%
% tikz settings

\usepackage{tikz}
\tikzset{every picture/.style={line width=0.75pt}}

%%%%%%%%%%%%%%%%%%%%%%%%%%%%

\usepackage{qrcode}



\title{Manipulator equations, Linearization}
\subtitle{Control Theory, Lecture 12}
\author{by Sergei Savin}
\centering
\date{Spring 2021}



\begin{document}
\maketitle


\begin{frame}{Content}
\begin{itemize}
\item Newton eq., Lagrange eq.
\item Manipulator eq.
\item \item 
\begin{itemize}
    \item Idea
    \item Linearization
\end{itemize}
\item Linearization of manipulator eq.
\begin{itemize}
    \item Change of variables
    \item State matrix
    \item Control matrix
\end{itemize}
\end{itemize}
\end{frame}


\begin{frame}{Newton eq., Lagrange eq.}
% \framesubtitle{Definitions}
\begin{flushleft}

In classical mechanics, \emph{Newton equations} are accepted as an axiom and are given (for a system of points):

\begin{equation}
\label{eq:Newton}
    m_i\ddot{\bo{r}}_i = \bo{f}_i, \ \ i = 1, \ ... \ m  
\end{equation}

where $m_i$ are masses of the particles, $\bo{r}_i$ are their positions, and $\bo{f}_i$ are forces acting on them.

\bigskip

From it, there is a way to derive \emph{Lagrange equations}:

\begin{equation}
\label{eq:Lagrange}
    \frac{d}{dt} \bigg( 
    \frac{\partial T }{\partial \dot{\bo{q}}}
    \bigg) - 
    \frac{\partial T }{\partial \bo{q}} = \tau
\end{equation}

where kinetic energy $T$ is defined as: $T = 0.5 \dot{\bo{r}} ^\top \bo{M} \dot{\bo{r}}$, $\bo{q}$ are generalized coordinates, and $\tau$ are generalized forces.

\end{flushleft}
\end{frame}




\begin{frame}{Manipulator eq.}
% \framesubtitle{Definitions}
\begin{flushleft}

Lagrange equations are useful when modelling a system of rigid bodies connected via \emph{joints}. Examples include robot arms, walking robots and others. However, there is an even more useful form of these equations, called \emph{manipulator equations}:

\begin{equation}
    \bo{H} \ddot{\bo{q}} + \bo{C} \dot{\bo{q}} + \bo{g} = \tau_n
\end{equation}

where $\bo{H} = \bo{H}(\bo{q})$ is a generalized inertia matrix, $\bo{C} = \bo{C}(\bo{q}, \dot{\bo{q}})$ is inertial force matrix, $\bo{g} = \bo{g}(\bo{q})$ is generalized gravity and other conservative forces, and $\tau_n$ are generalized non-conservative forces, such as control inputs (motor torques, etc.)

\end{flushleft}
\end{frame}




\begin{frame}{Linear control for nonlinear systems}
\framesubtitle{Idea}
\begin{flushleft}

Consider the following problem: given $\bo{H} \ddot{\bo{q}} + \bo{C} \dot{\bo{q}} + \bo{g} = \tau_n$, design control law that would stabilize the system at the origin.

\bigskip

\begin{block}{Linearization}
One of the ways to think about this problem is to imagine that if there existed a linear system, that behaves exactly like the original system in the vicinity of the origin, we could stabilize the linear one and then use the found control law to stabilize the non-linear version.
\end{block}


\end{flushleft}
\end{frame}




\begin{frame}{Linear control for nonlinear systems}
\framesubtitle{Linearization}
\begin{flushleft}

In general, for a system of the form $\dot{\bo{x}} = \bo{f}(\bo{x}, \bo{u})$, there is a simple way to linearize it around any given point. The process if based on Taylor expansion:

\begin{equation}
    \bo{f}(\bo{x}, \bo{u}) 	\sim \bo{f}(\bo{x}_0, \bo{u}_0) +
    \frac{\partial \bo{f}}{\partial \bo{x}} (\bo{x} - \bo{x}_0) + 
    \frac{\partial \bo{f}}{\partial \bo{u}} (\bo{u} - \bo{u}_0)
\end{equation}

Denoting $\frac{\partial \bo{f}}{\partial \bo{x}} = \bo{A}$ and $\frac{\partial \bo{f}}{\partial \bo{u}} = \bo{B}$, we get:

\begin{equation}
    \bo{f}(\bo{x}, \bo{u}) 	\sim \bo{f}(\bo{x}_0, \bo{u}_0) +
    \bo{A} (\bo{x} - \bo{x}_0) + 
    \bo{B} (\bo{u} - \bo{u}_0)
\end{equation}

Finally, denoting $ \bo{f}(\bo{x}_0, \bo{u}_0) - \bo{A}\bo{x}_0 - \bo{B} \bo{u}_0 = \bo{c}$ we achieve local linearization of the nonlinear dynamics:

\begin{equation}
    \dot{\bo{x}} \sim \bo{A} \bo{x} + 
    \bo{B} \bo{u} + \bo{c}
\end{equation}


\end{flushleft}
\end{frame}



\begin{frame}{Linear control for nonlinear systems}
\framesubtitle{Linearization of manipulator eq,}
\begin{flushleft}

We, on the other hand, have eq. in the form $\bo{H} \ddot{\bo{q}} + \bo{C} \dot{\bo{q}} + \bo{g} = \tau_n$. Let us assume $\tau_n = \bo{T}\bo{u}$, where $\bo{T} = \bo{T}(\bo{q})$. We can also define $\bo{x}$:

\begin{equation}
    \bo{x} = \begin{bmatrix}
    \bo{q} \\ \dot{\bo{q}} \end{bmatrix}
\end{equation}

Then our attempt to write system of first order diff. eq. gives us:

\begin{equation}
    \frac{d}{dt} 
    \left(
    \begin{bmatrix}
    \bo{q} \\ \dot{\bo{q}} 
    \end{bmatrix}
    \right)
    =
    \begin{bmatrix}
    \dot{\bo{q}}  \\ 
    \bo{H}^{-1} (\bo{T}\bo{u} -  \bo{C} \dot{\bo{q}} - \bo{g}) 
    \end{bmatrix}
\end{equation}

\end{flushleft}
\end{frame}




\begin{frame}{Linear control for nonlinear systems}
\framesubtitle{State matrix}
\begin{flushleft}

In this case, state matrix $\bo{A}$ becomes:

\begin{equation}
    \bo{A} = 
    \begin{bmatrix}
    0 & \bo{I} \\
    \frac{\partial (\bo{H}^{-1} (\bo{T}\bo{u} -  \bo{C} \dot{\bo{q}} - \bo{g}) )}{\partial \bo{q}} 
    &
    \frac{\partial (\bo{H}^{-1} (\bo{T}\bo{u} -  \bo{C} \dot{\bo{q}} - \bo{g}) )}{\partial \dot{\bo{q}}}
    \end{bmatrix}
\end{equation}

Let us expand:

\begin{equation}
    \frac{\partial }{\partial \bo{q}} (\bo{H}^{-1} (\bo{T}\bo{u} -  \bo{C} \dot{\bo{q}} - \bo{g}) ) 
    = 
    \frac{\partial \bo{H}^{-1}}{\partial \bo{q}} (\bo{T}\bo{u} -  \bo{C} \dot{\bo{q}} - \bo{g})
    + 
    \bo{H}^{-1} 
    \frac{\partial }{\partial \bo{q}} 
    (\bo{T}\bo{u} -  \bo{C} \dot{\bo{q}} - \bo{g}) )
\end{equation}

\begin{equation}
    \frac{\partial \bo{H}^{-1}}{\partial \bo{q}} = 
    - \bo{H}^{-1} \frac{\partial \bo{H}}{\partial \bo{q}} 
    \bo{H}^{-1}
\end{equation}

\end{flushleft}
\end{frame}




\begin{frame}{Linear control for nonlinear systems}
\framesubtitle{State matrix, Control matrix}
\begin{flushleft}

...and:

\begin{equation}
\frac{\partial (\bo{H}^{-1} (\bo{T}\bo{u} -  \bo{C} \dot{\bo{q}} - \bo{g}) )}{\partial \dot{\bo{q}}} 
= 
-\bo{H}^{-1}\bo{C} - \bo{H}^{-1} 
\frac{\partial \bo{C}}{\partial \dot{\bo{q}}}  \dot{\bo{q}}
\end{equation}

Control matrix $\bo{B}$ becomes:

\begin{equation}
    \bo{B} = 
    \begin{bmatrix}
    0 & 0 \\
    \frac{\partial (\bo{H}^{-1} (\bo{T}\bo{u} -  \bo{C} \dot{\bo{q}} - \bo{g}) )}{\partial \bo{u}} 
    &
    0
    \end{bmatrix}
    =
        \begin{bmatrix}
    0 & 0 \\
    \bo{H}^{-1} \bo{T}
    &
    0
    \end{bmatrix}
\end{equation}

...and this does not look very clean and nice to use. Indeed, it is not easy or nice in practice.

\end{flushleft}
\end{frame}



\begin{frame}{Thank you!}
\centerline{Lecture slides are available via Moodle.}
\bigskip
\centerline{You can help improve these slides at:}
\centerline{
\textcolor{blue}{\centerline{\href{https://github.com/SergeiSa/Control-Theory-Slides-Spring-2021}{github.com/SergeiSa/Control-Theory-Slides-Spring-2021}}}}

\bigskip

\textcolor{black}{\qrcode[height=1.5in]{github.com/SergeiSa/Control-Theory-Slides-Spring-2021}}

\bigskip

\centerline{Check Moodle for additional links, videos, textbook suggestions.}
\end{frame}

\end{document}
