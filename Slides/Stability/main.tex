\documentclass{beamer}

\input{settings.tex}


\title{Stability}
\subtitle{Control Theory, Lecture 2}
\author{by Sergei Savin}
\centering
\date{\mydate}



\begin{document}
\maketitle


\begin{frame}{Content}

\begin{itemize}
\item Critical point (node)
\item Stability
\item Asymptotic stability
\item Stability vs Asymptotic stability
\item LTI and autonomous LTI
\item Stability of autonomous LTI
    \begin{itemize}
    \item Example: real eigenvalues
    \item Example: complex eigenvalues
    \item General case
    \item Illustration
    \end{itemize}
\item Read more
\end{itemize}

\end{frame}



\begin{frame}{Critical point (node)}
% \framesubtitle{O}
\begin{flushleft}

Consider the following ODE:

\begin{equation}
    \dot{\bo{x}} = \bo{f} (\bo{x}, t)
\end{equation}

Let $\bo{x}_0$ be such a state that:

\begin{equation}
    \bo{f} (\bo{x}_0, t) = 0
\end{equation}

Then such state $\bo{x}_0$ is called a \emph{node} or a \emph{critical point}.

\end{flushleft}
\end{frame}



\begin{frame}{Stability}
% \framesubtitle{O}
\begin{flushleft}

Node $\bo{x}_0$ is called \emph{stable} iff for any constant $\delta$ there exists constant $\varepsilon$ such that:

\begin{equation}
    ||\bo{x}(0) - \bo{x}_0|| < \delta \ \longrightarrow \ ||\bo{x}(t) - \bo{x}_0|| < \varepsilon
\end{equation}

\bigskip

Think of it as "for any initial point that lies at most $\delta$ away from $\bo{x}_0$, the rest of the trajectory $\bo{x}(t)$ will be at most $\varepsilon$ away from $\bo{x}_0$".

\bigskip

Or, more picturesque, think of it as "the solutions with different initial conditions do not diverge from the node"

\end{flushleft}
\end{frame}


\begin{frame}{Asymptotic stability}
% \framesubtitle{O}
\begin{flushleft}

Node $\bo{x}_0$ is called \emph{asymptotically stable} iff for any constant $\delta$ it is true that:

\begin{equation}
    ||\bo{x}(0) - \bo{x}_0|| < \delta \ \longrightarrow \ 
    \lim_{t\to\infty} \bo{x}(t) = \bo{x}_0
\end{equation}

\bigskip

Think of it as "for any initial point that lies at most $\delta$ away from $\bo{x}_0$, the trajectory $\bo{x}(t)$ will asymptotically approach the point $\bo{x}_0$".

\bigskip

Or, more picturesque, think of it as "the solutions with different initial conditions converge to the node"

\end{flushleft}
\end{frame}




\begin{frame}{Stability vs Asymptotic stability}
% \framesubtitle{O}
\begin{flushleft}

\begin{example}
Consider dynamical system $\dot{x} = 0$, and solution $x = 7$. This solution is stable, but not asymptotically stable (other solutions do not diverge from $x = 7$, but do not converge to it either).
\end{example}

\begin{example}
Consider dynamical system $\dot{x} = -x$, and solution $x = 0$. This solution is stable and asymptotically stable (other solutions converge to $x = 0$).
\end{example}

\begin{example}
Consider dynamical system $\dot{x} = x$, and solution $x = 0$. This solution is unstable (other solutions diverge from $x = 0$).
\end{example}

\end{flushleft}
\end{frame}



\begin{frame}{LTI and autonomous LTI}
% \framesubtitle{O}
\begin{flushleft}

Consider the following linear ODE:

\begin{equation}
    \dot{\bo{x}} = \bo{A} \bo{x} + \bo{B} \bo{u}
\end{equation}

This is called a \emph{linear time-invariant system}, or \emph{LTI}.

\bigskip

Consider the following linear ODE:

\begin{equation}
    \dot{\bo{x}} = \bo{A} \bo{x}
\end{equation}

This is also an LTI, but it is also called an \emph{autonomous system}, since its evolution depends only on the state of the system.

\end{flushleft}
\end{frame}




\begin{frame}{Stability of autonomous LTI}
\framesubtitle{Real eigenvalues}
\begin{flushleft}

Consider autonomous LTI:

\begin{equation}
    \dot{\bo{x}} = \bo{A} \bo{x}
\end{equation}

where $\bo{A}$ can be decomposed via eigen-decomposition as $\bo{A} = \bo{V} \bo{D} \bo{V}^{-1}$, where $\bo{D}$ is a diagonal matrix. 

\bigskip

\begin{equation}
    \dot{\bo{x}} = \bo{V} \bo{D} \bo{V}^{-1} \bo{x}
\end{equation}

Multiply it by $\bo{V}^{-1} 
\ \longrightarrow \ 
\bo{V}^{-1} \dot{\bo{x}} = \bo{V}^{-1} \bo{V} \bo{D} \bo{V}^{-1} \bo{x}$.

Define $\bo{z} = \bo{V}^{-1} \bo{x} 
\ \longrightarrow \
\dot{\bo{z}} = \bo{D} \bo{z}$.

\bigskip

Since elements of $\bo{D}$ are real, we can clearly see, that iff they are \emph{all negative} will the system be asymptotically stable. If they are non-positive, the system is stable. And those elements are eigenvalues of $\bo{A}$.

\end{flushleft}
\end{frame}



\begin{frame}{Stability of autonomous LTI}
\framesubtitle{Complex eigenvalues, 2-dimensional case (1)}
\begin{flushleft}

Let us consider the following system:

\begin{equation}
\begin{bmatrix}
    \dot{\bo{x}}_1 \\ \dot{\bo{x}}_2
\end{bmatrix}
     = 
\begin{bmatrix}
    \alpha & -\beta \\ \beta & \alpha
\end{bmatrix}     
\begin{bmatrix}
    \bo{x}_1 \\ \bo{x}_2
\end{bmatrix}
\end{equation}

The eigenvalues of the system are $\alpha \pm i \beta$. We denote $\begin{bmatrix}
    \bo{x}_1 \\ \bo{x}_2
\end{bmatrix} = \bo{x}$.

\bigskip

We start by claiming that the system will be stable iff the $\dot{\bo{x}}^\top \bo{x} < 0$. Indeed, vector $\dot{\bo{x}}$ can always be decomposed into two components, $\dot{\bo{x}}_{||}$ parallel to $\bo{x}$, and $\dot{\bo{x}}_{\perp}$ perpendicular to $\bo{x}$. By definition $\dot{\bo{x}}_{\perp}^\top \bo{x} = 0$, and is responsible for the change in orientation of $\bo{x}$. The value of $\dot{\bo{x}}_{||}$ is responsible for the change in the length of $\bo{x}$; the length would shrink iff $\dot{\bo{x}}_{||}$ is of opposite direction to $\bo{x}$, giving negative value of the dot product $\dot{\bo{x}}^\top \bo{x}$.

\end{flushleft}
\end{frame}



\begin{frame}{Stability of autonomous LTI}
\framesubtitle{Complex eigenvalues, 2-dimensional case (2)}
\begin{flushleft}

Let us compute $\dot{\bo{x}}^\top \bo{x}$:

\begin{equation}
\dot{\bo{x}}^\top \bo{x} =
\begin{bmatrix}
    \bo{x}_1 & \bo{x}_2
\end{bmatrix}
\begin{bmatrix}
    \alpha & -\beta \\ \beta & \alpha
\end{bmatrix}     
\begin{bmatrix}
    \bo{x}_1 \\ \bo{x}_2
\end{bmatrix}
\end{equation}

\begin{equation}
\dot{\bo{x}}^\top \bo{x} =
\alpha (\bo{x}_1^2 + \bo{x}_2^2)
\end{equation}

From this it is clear that the product $\dot{\bo{x}}^\top \bo{x} < 0$ is negative iff $\alpha < 0$.

\begin{definition}
As long as the \emph{real parts of the eigenvalues} of the system are \emph{strictly negative}, the system is \emph{asymptotically stable}. If the real parts of the eigenvalues of the system are zero, the system is \emph{marginally stable}.
\end{definition}

\end{flushleft}
\end{frame}



\begin{frame}{Stability of autonomous LTI}
\framesubtitle{Complex eigenvalues, 2-dimensional case (3)}
\begin{flushleft}

Vector field of 
$\begin{bmatrix}
    \dot{\bo{x}}_1 \\ \dot{\bo{x}}_2
\end{bmatrix} 
=
\begin{bmatrix}
    \alpha & -\beta \\ \beta & \alpha
\end{bmatrix}     
\begin{bmatrix}
    \bo{x}_1 \\ \bo{x}_2
\end{bmatrix} $ 
is shown below:
%
\begin{figure}
    \centering
    \includegraphics[width=7cm]{Figure_1.png}%, width=7cm
    % \caption{Caption}
    \label{fig:my_label}
\end{figure}

\end{flushleft}
\end{frame}



\begin{frame}{Stability of autonomous LTI}
\framesubtitle{General case (1)}
\begin{flushleft}

Given $\dot{\bo{x}} = \bo{A} \bo{x}$, where $\bo{A}$ can be decomposed via eigen-decomposition as $\bo{A} = \bo{U} \bo{C} \bo{U}^{-1}$, where $\bo{C}$ is a complex-valued diagonal matrix and $\bo{U}$ is a complex-valued inevitable matrix. 

\bigskip

We multiply both sides by $\bo{U}^{-1}$, then define $\bo{z} = \bo{U}^{-1} \bo{x}$ to arrive at:

\begin{equation}
    \dot{\bo{z}} = \bo{C} \bo{z}
\end{equation}

which falls into a set of independent equations, with complex coefficients $c_j$:

\begin{equation}
    \dot{z}_j = c_j z_j
\end{equation}

\end{flushleft}
\end{frame}



\begin{frame}{Stability of autonomous LTI}
\framesubtitle{General case (2)}
\begin{flushleft}

Expanding $c_j = \alpha + i \beta$, and $z_j = u + i v$ (we dismiss subscripts for clarity), we find that $\dot{z}_j = c_j z_j$ can be expanded as:

\begin{equation}
    \dot{u} + i \dot{v} = \dot{z}_j = c_j z_j = (\alpha + i \beta) (u + i v)
\end{equation}
%
\begin{equation}
    \dot{u} + i \dot{v} = \alpha u + i \beta u + i \alpha v - \beta v
\end{equation}
%
\begin{equation}
\begin{bmatrix}
    \dot{u} \\ \dot{v}
\end{bmatrix}
     = 
\begin{bmatrix}
    \alpha & -\beta \\ \beta & \alpha
\end{bmatrix}     
\begin{bmatrix}
    u \\ v
\end{bmatrix}
\end{equation}

As we can see, $\dot{z}_j = c_j z_j$ is asymptotically stable iff $\text{Re}(c_j) < 0$, and marginally stable if $\alpha = \text{Re}(c_j) = 0$. Same is true for $\dot{\bo{z}} = \bo{C} \bo{z}$ and hence, for $\dot{\bo{x}} = \bo{A} \bo{x}$, as $\bo{U}$ is invertible.

\end{flushleft}
\end{frame}




\begin{frame}{Stability of autonomous LTI}
\framesubtitle{Condition}
\begin{flushleft}

Consider an autonomous LTI:

\begin{equation}
\label{eq:LTI}
    \dot{\bo{x}} = \bo{A} \bo{x}
\end{equation}

\begin{definition}
Eq. \eqref{eq:LTI} is stable iff real parts of eigenvalues of $\bo{A}$ are non-positive.
\end{definition}

\begin{definition}
Eq. \eqref{eq:LTI} is asymptotically stable iff real parts of eigenvalues of $\bo{A}$ are negative.
\end{definition}

\end{flushleft}
\end{frame}




\begin{frame}{Stability of autonomous LTI}
\framesubtitle{Illustration}
\begin{flushleft}

Here is an illustration of \emph{phase portraits} of two-dimensional LTIs with different types of stability:

\begin{figure}
    \centering
    \includegraphics[width=1.0\linewidth]{Stability.PNG}
    \caption{phase portraits for different types of stability}
    \label{fig:Stability}
\end{figure}

\bigskip

\scriptsize{Credit: \bref{http://staff.uz.zgora.pl/wpaszke/materialy/spc/Lec13.pdf}{staff.uz.zgora.pl/wpaszke/materialy/spc/Lec13.pdf}}

\end{flushleft}
\end{frame}


\begin{frame}
\hspace*{-2.5cm}
\includegraphics[height=\textheight,width=1.4\textwidth,keepaspectratio]{Figure_2.png}
\end{frame}



\begin{frame}{Read more}

\begin{itemize}
\item Control Systems Design, by Julio H. Braslavsky \bref{http://staff.uz.zgora.pl/wpaszke/materialy/spc/Lec13.pdf}{staff.uz.zgora.pl/wpaszke/materialy/spc/Lec13.pdf}


\end{itemize}

\end{frame}



\begin{frame}{Thank you!}
\centerline{Lecture slides are available via Moodle.}
\bigskip
\centerline{You can help improve these slides at:}
\centerline{\mygit}
\bigskip
\centerline{Check Moodle for additional links, videos, textbook suggestions.}
\bigskip

\centerline{\textcolor{black}{\qrcode[height=1.6in]{https://github.com/SergeiSa/Control-Theory-Slides-Spring-2022}}}
\end{frame}

\end{document}
