\documentclass{beamer}

\pdfmapfile{+sansmathaccent.map}


\mode<presentation>
{
  \usetheme{Warsaw} % or try Darmstadt, Madrid, Warsaw, Rochester, CambridgeUS, ...
  \usecolortheme{seahorse} % or try seahorse, beaver, crane, wolverine, ...
  \usefonttheme{serif}  % or try serif, structurebold, ...
  \setbeamertemplate{navigation symbols}{}
  \setbeamertemplate{caption}[numbered]
} 


%%%%%%%%%%%%%%%%%%%%%%%%%%%%
% itemize settings

\definecolor{mypink}{RGB}{255, 30, 80}
\definecolor{mydarkblue}{RGB}{60, 160, 255}
\definecolor{myblue}{RGB}{240, 240, 255}
\definecolor{mygreen}{RGB}{0, 200, 0}
\definecolor{mygreen2}{RGB}{245, 255, 230}
\definecolor{mygray}{gray}{0.8}

\setbeamertemplate{itemize items}[default]

\setbeamertemplate{itemize item}{\color{mygreen}$\blacksquare$}
\setbeamertemplate{itemize subitem}{\color{mydarkblue}$\blacktriangleright$}
\setbeamertemplate{itemize subsubitem}{\color{mygray}$\blacksquare$}



\setbeamercolor{palette quaternary}{fg=white,bg=mydarkblue}
\setbeamercolor{titlelike}{parent=palette quaternary}

\setbeamercolor{palette quaternary2}{fg=black,bg=myblue}
\setbeamercolor{frametitle}{parent=palette quaternary2}



\setbeamerfont{frametitle}{size=\Large,series=\scshape}
\setbeamerfont{framesubtitle}{size=\normalsize,series=\upshape}





%%%%%%%%%%%%%%%%%%%%%%%%%%%%
% block settings

\setbeamercolor{block title}{bg=red!30,fg=black}

\setbeamercolor*{block title example}{bg=mygreen!40!white,fg=black}

\setbeamercolor*{block body example}{fg= black,
bg= mygreen2}


%%%%%%%%%%%%%%%%%%%%%%%%%%%%
% URL settings
\hypersetup{
    colorlinks=false,
    linkcolor=blue,
    filecolor=blue,      
    urlcolor=blue,
}

%%%%%%%%%%%%%%%%%%%%%%%%%%

\renewcommand{\familydefault}{\rmdefault}

\usepackage{amsmath}
\usepackage{mathtools}


\usepackage{subcaption}

\newcommand{\bo}[1] {\mathbf{#1}}
\newcommand{\R} {\mathbb{R}}
\DeclareMathOperator*{\argmin}{arg\,min}


%%%%%%%%%%%%%%%%%%%%%%%%%%%%
% code settings

\usepackage{listings}
\usepackage{color}
% \definecolor{mygreen}{rgb}{0,0.6,0}
% \definecolor{mygray}{rgb}{0.5,0.5,0.5}
\definecolor{mymauve}{rgb}{0.58,0,0.82}
\lstset{ 
  backgroundcolor=\color{white},   % choose the background color; you must add \usepackage{color} or \usepackage{xcolor}; should come as last argument
  basicstyle=\footnotesize,        % the size of the fonts that are used for the code
  breakatwhitespace=false,         % sets if automatic breaks should only happen at whitespace
  breaklines=true,                 % sets automatic line breaking
  captionpos=b,                    % sets the caption-position to bottom
  commentstyle=\color{mygreen},    % comment style
  deletekeywords={...},            % if you want to delete keywords from the given language
  escapeinside={\%*}{*)},          % if you want to add LaTeX within your code
  extendedchars=true,              % lets you use non-ASCII characters; for 8-bits encodings only, does not work with UTF-8
  firstnumber=0000,                % start line enumeration with line 0000
  frame=single,	                   % adds a frame around the code
  keepspaces=true,                 % keeps spaces in text, useful for keeping indentation of code (possibly needs columns=flexible)
  keywordstyle=\color{blue},       % keyword style
  language=Octave,                 % the language of the code
  morekeywords={*,...},            % if you want to add more keywords to the set
  numbers=left,                    % where to put the line-numbers; possible values are (none, left, right)
  numbersep=5pt,                   % how far the line-numbers are from the code
  numberstyle=\tiny\color{mygray}, % the style that is used for the line-numbers
  rulecolor=\color{black},         % if not set, the frame-color may be changed on line-breaks within not-black text (e.g. comments (green here))
  showspaces=false,                % show spaces everywhere adding particular underscores; it overrides 'showstringspaces'
  showstringspaces=false,          % underline spaces within strings only
  showtabs=false,                  % show tabs within strings adding particular underscores
  stepnumber=2,                    % the step between two line-numbers. If it's 1, each line will be numbered
  stringstyle=\color{mymauve},     % string literal style
  tabsize=2,	                   % sets default tabsize to 2 spaces
  title=\lstname                   % show the filename of files included with \lstinputlisting; also try caption instead of title
}

%%%%%%%%%%%%%%%%%%%%%%%%%%%%
% tikz settings

\usepackage{tikz}
\tikzset{every picture/.style={line width=0.75pt}}

%%%%%%%%%%%%%%%%%%%%%%%%%%%%

\usepackage{qrcode}



\title{Controllability, Observability}
\subtitle{Control Theory, Lecture 11}
\author{by Sergei Savin}
\centering
\date{\mydate}



\begin{document}
\maketitle


\begin{frame}{Content}
\begin{itemize}
\item Controllability of Discrete LTI
\begin{itemize}
    \item Controllability matrix
    \item Controllability criterion
\end{itemize}
\item Observability of Discrete LTI
\begin{itemize}
    \item Dual system
    \item Observability criterion
\end{itemize}
\item ”Unlimited control”
\item Limited control
\end{itemize}
\end{frame}



\begin{frame}{Controllability}
	% \framesubtitle{Definitions}
	\begin{flushleft}
		
		
		
	\end{flushleft}
\end{frame}


\begin{frame}{Controllability of Discrete LTI}
% \framesubtitle{Definitions}
\begin{flushleft}

Consider discrete LTI:
\begin{equation}
\bo{x}_{i+1} = \bo{A}  \bo{x}_i + \bo{B} \bo{u}_i
\end{equation}

Assume the initial state is $\bo{x}_1$. Then we can deduce that:

\begin{align*}
\bo{x}_2 &= \bo{A} \bo{x}_1 + \bo{B} \bo{u}_1 \\
\bo{x}_3 &= \bo{A} \bo{x}_2 + \bo{B} \bo{u}_2 = \bo{A} (\bo{A} \bo{x}_1 + \bo{B} \bo{u}_1) + \bo{B} \bo{u}_2 \\
\bo{x}_4 &= \bo{A} \bo{x}_3 + \bo{B} \bo{u}_3 = \bo{A} (\bo{A} (\bo{A} \bo{x}_1 + \bo{B} \bo{u}_1) + \bo{B} \bo{u}_2) + \bo{B} \bo{u}_3 \\
... \\
\bo{x}_{n+1} &= \bo{A}^n \bo{x}_1 + ... + 
\bo{A}^{n - k} \bo{B} \bo{u}_{k} + ... + 
\bo{B} \bo{u}_n
\end{align*}

\end{flushleft}
\end{frame}



\begin{frame}{Controllability of Discrete LTI}
\framesubtitle{Controllability matrix}
\begin{flushleft}

Equation $\bo{x}_{n+1} = \bo{A}^n \bo{x}_1 + ... 
+ \bo{A}^{n - k} \bo{B} \bo{u}_{k} + ...
\bo{B} \bo{u}_n$ can be re-written as:

\begin{equation}
    \bo{x}_{n+1} - \bo{A}^n \bo{x}_1 = 
    \begin{bmatrix}
    \bo{B} &
    \bo{A} \bo{B} &
    \bo{A}^2 \bo{B} & ... &
    \bo{A}^{n - 1} \bo{B}
    \end{bmatrix}    
    \begin{bmatrix}
    \bo{u}_{k} \\
    \bo{u}_{k-1} \\
    \bo{u}_{k-2} \\ ... \\
    \bo{u}_{1}
    \end{bmatrix}
\end{equation}

Notice that in order for the system to go from $\bo{x}_1$ to $\bo{x}_{n+1}$, vector $\bo{x}_{n+1} - \bo{A}^n \bo{x}_1$ needs be in the column space of $\mathcal{C} = \begin{bmatrix}
    \bo{B} &
    \bo{A} \bo{B} & ... &
    \bo{A}^{n - 1} \bo{B}
    \end{bmatrix}$.

Since $\bo{x}_{n+1}$ can be anything, and $\bo{x}_1$ might be equal to zero (among other possibilities), we should require that all vectors in $\R^n$ need to be in the column space of $\mathcal{C}$, meaning $\mathcal{C}$ needs to be full rank.

\end{flushleft}
\end{frame}


\begin{frame}{Controllability of Discrete LTI}
\framesubtitle{Controllability criterion}
\begin{flushleft}

\begin{block}{Controllability}
For a system $\bo{x}_{i+1} = \bo{A}  \bo{x}_i + \bo{B} \bo{u}_i$, where $\bo{x} \in \R^n$, if the matrix $\mathcal{C} = \begin{bmatrix}
    \bo{B} &
    \bo{A} \bo{B} & ... &
    \bo{A}^{n - 1} \bo{B}
    \end{bmatrix}$ is full row rank (i.e. $\text{rank}(\mathcal{C}) = n$), any state can be reached, which means that \emph{the system is controllable}.
\end{block}

If you are interested why the controllability matrix for not include more columns, like $\bo{A}^n$, see Appendix A.

\end{flushleft}
\end{frame}




\begin{frame}{Observability of Discrete LTI}
% \framesubtitle{Definitions}
\begin{flushleft}

Consider discrete LTI:
\begin{equation}
\begin{cases}
\bo{x}_{i+1} = \bo{A}  \bo{x}_i + \bo{B} \bo{u}_i \\
\bo{y}_i     = \bo{C}  \bo{x}_i
\end{cases}
\end{equation}

And an observer:

\begin{equation}
\hat{\bo{x}}_{i+1} = \bo{A}  \hat{\bo{x}}_i + \bo{B} \bo{u}_i + 
\bo{L} (\bo{y}_i - \bo{C} \hat{\bo{x}}_i)
\end{equation}

Remember that we can define observation error $\bo{e}_i = \hat{\bo{x}}_i - \bo{x}_i$ and write its dynamics:

\begin{equation}
\bo{e}_{i+1} = \bo{A} \bo{e}_i - 
\bo{L} \bo{C} \bo{e}_i
\end{equation}

Dual system (which is stable if and only if the original is stable), has form:

\begin{equation}
\varepsilon_{i+1} = \bo{A}^\top \varepsilon_i - 
\bo{C}^\top \bo{L}^\top \varepsilon_i
\end{equation}


\end{flushleft}
\end{frame}




\begin{frame}{Observability of Discrete LTI}
\framesubtitle{Dual system}
\begin{flushleft}

Dynamical system $\varepsilon_{i+1} = \bo{A}^\top \varepsilon_i - \bo{C}^\top \bo{L}^\top \varepsilon_i$, we can be represented as:

\begin{equation}
\begin{cases}
\varepsilon_{i+1} = \bo{A}^\top \varepsilon_i + \bo{C}^\top \bo{v}_i \\
\bo{v}_i = - \bo{L}^\top \varepsilon_i
\end{cases}
\end{equation}

Controllability matrix of this system is:

\begin{equation}
\mathcal{O}^\top = \begin{bmatrix}
    \bo{C}^\top &
    (\bo{A}^\top) \bo{C}^\top & ... &
    (\bo{A}^\top)^{n - 1} \bo{C}^\top
    \end{bmatrix}
\end{equation}

It is easier to represent this matrix in its transposed form:

\begin{equation}
\mathcal{O} = \begin{bmatrix}
    \bo{C} \\
    \bo{C}\bo{A}  \\ ... \\
    \bo{C}\bo{A}^{n - 1}
    \end{bmatrix}
\end{equation}

\end{flushleft}
\end{frame}


\begin{frame}{Observability of Discrete LTI}
\framesubtitle{Observability criterion}
\begin{flushleft}

\begin{block}{Observability}
For a system $\bo{x}_{i+1} = \bo{A}  \bo{x}_i + \bo{B} \bo{u}_i$ and $\bo{y}_i = \bo{C}  \bo{x}_i$, where $\bo{x} \in \R^n$, if the matrix $\mathcal{O} = \begin{bmatrix}
    \bo{C} \\
    \bo{C}\bo{A}  \\ ... \\
    \bo{C}\bo{A}^{n - 1}
    \end{bmatrix}$ is full column rank (i.e. $\text{rank}(\mathcal{O}) = n$), observation error can go to zero from any initial position, which means that \emph{the system is observable}.
\end{block}

\end{flushleft}
\end{frame}





\begin{frame}{Control}
\framesubtitle{"Unlimited control", part 1}
\begin{flushleft}

Let's look at this equation one more time:

\begin{equation}
    \bo{x}_{n+1} - \bo{A}^n \bo{x}_1 = 
    \begin{bmatrix}
    \bo{B} &
    \bo{A} \bo{B} &
    \bo{A}^2 \bo{B} & ... &
    \bo{A}^{n - 1} \bo{B}
    \end{bmatrix}    
    \begin{bmatrix}
    \bo{u}_{k} \\
    \bo{u}_{k-1} \\
    \bo{u}_{k-2} \\ ... \\
    \bo{u}_{1}
    \end{bmatrix}
\end{equation}

If the system is controllable, it means \emph{every state can be reached from any other space in only $n$ steps}. This seem to disagree with our real-world experience.

\end{flushleft}
\end{frame}




\begin{frame}{Control}
\framesubtitle{"Unlimited control", part 2}
\begin{flushleft}

Let's look at an even simpler equation $\bo{x}_{i+1} = \bo{A}  \bo{x}_i + \bo{B} \bo{u}_i$. Let's rewrite the equation as follows:

\begin{equation}
\bo{x}_f - \bo{A} \bo{x}_1 = \bo{B} \bo{u}_1
\end{equation}

As long as $\bo{x}_f - \bo{A} \bo{x}_1$ lies in the column space of $\bo{B}$, it \emph{can be achieved in a single step}, using control:

\begin{equation}
\bo{u}_1 = \bo{B}^+ (\bo{x}_f - \bo{A} \bo{x}_1)
\end{equation}

This as well, seem to disagree with our real-world experience.

\end{flushleft}
\end{frame}



\begin{frame}{Limited control}
% \framesubtitle{Limited control}
\begin{flushleft}

In the actual engineering reality we often have to deal with equations, that look closer to:

\begin{equation}
\begin{cases}
\bo{x}_{i+1} = \bo{A}  \bo{x}_i + \bo{B} \bo{u}_i \\
|| \bo{D} \bo{u}_i ||_r \leq 1
\end{cases}
\end{equation}

... which is a \emph{second-order cone program}. Or:

\begin{equation}
\begin{cases}
\bo{x}_{i+1} = \bo{A}  \bo{x}_i + \bo{B} \bo{u}_i \\
\bo{D} \bo{u}_i \leq \bo{d}
\end{cases}
\end{equation}

... which is a \emph{quadratic program}. Notice, those equations \emph{can't be solved analytically}.

\end{flushleft}
\end{frame}





\begin{frame}{Thank you!}
	\centerline{Lecture slides are available via Moodle.}
	\bigskip
	\centerline{You can help improve these slides at:}
	\centerline{\mygit}
	\bigskip
	\centerline{Check Moodle for additional links, videos, textbook suggestions.}
	\bigskip
	
	\centerline{\textcolor{black}{\qrcode[height=1.6in]{https://github.com/SergeiSa/Control-Theory-Slides-Spring-2022}}}
\end{frame}



\begin{frame}{Appendix A, part 1}
	% \framesubtitle{Limited control}
	\begin{flushleft}
		
		Why does controllability matrix $\mathcal{C} = \begin{bmatrix}
			\bo{B} &
			\bo{A} \bo{B} & ... &
			\bo{A}^{n - 1} \bo{B}
		\end{bmatrix}$  includes only column blocks up to $\bo{A}^{n - 1} \bo{B}$ and not, for example, $\bo{A}^n \bo{B}$? We start with:
	
		\begin{theorem}[Cayley–Hamilton]
			A matrix $\bo{M} \in \R^{n, n}$ satisfies its own characteristic equation.
		\end{theorem}
	
		A characteristic equation can be written as $\lambda^n + a_{n-1}\lambda^{n-1} + ... + a_0  = 0$, meaning that we can write:
		
		\begin{equation}
			\bo{M}^n + a_{n-1}\bo{M}^{n-1} + ... + a_0\bo{I}  = 0
		\end{equation}	
		
		Meaning that $\bo{M}^n$ is a linear combination of $\bo{M}^{n-1}$, $\bo{M}^{n-2}$, ..., $\bo{I}$.
	
		
	\end{flushleft}
\end{frame}



\begin{frame}{Appendix A, part 2}
	% \framesubtitle{Limited control}
	\begin{flushleft}
		
		The controllability matrix can be written as 
		
		\begin{equation}
			\mathcal{C} = \begin{bmatrix}
				\bo{I} &
				\bo{A}  & ... &
				\bo{A}^{n - 1} 
			\end{bmatrix}
			\begin{bmatrix}
				\bo{B} & \bo{0} & ... & \bo{0} \\
				\bo{0}  & \bo{B} & ... & \bo{0} \\
				...  & ... & ... & ... \\
				\bo{0}  & \bo{0} & ... & \bo{B} 
			\end{bmatrix}
		\end{equation}
		
		meaning that the rank of $\mathcal{C}$ depends only on matrix $\begin{bmatrix}
			\bo{I} &
			\bo{A}  & ... &
			\bo{A}^{n - 1} 
		\end{bmatrix}$. Adding  to it columns $\bo{A}^n$ does not change the rank, as $\bo{A}^n$ is a linear combination of the other columns, as we proved in the previous slide.
		
	\end{flushleft}
\end{frame}





\end{document}
